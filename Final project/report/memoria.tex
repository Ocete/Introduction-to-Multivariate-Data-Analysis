\documentclass[11pt,a4paper]{article}

% Packages
\usepackage[utf8]{inputenc}
\usepackage[english]{babel}
\usepackage{caption}
\usepackage{listings}
\usepackage{adjustbox}
\usepackage{enumitem}
\usepackage{dsfont}
\usepackage{boldline}
\usepackage{amssymb, amsmath}
\usepackage[margin=1in]{geometry}
\usepackage{xcolor}
\usepackage{enumerate}
\usepackage{hyperref}
\usepackage{graphics, graphicx, float}
\usepackage{titlesec} %\titleformat

% Meta
\title{Introduction to Multivariate Data Analysis
	\\\medskip \large Final Project Report}
\author{José Antonio Álvarez Ocete - 917933752 \\ jocete@ucdavis.edu}
\date{ \today }

% Custom
\providecommand{\abs}[1]{\lvert#1\rvert}
\setlength\parindent{0pt}
\definecolor{Light}{gray}{.90}
\setlength{\parindent}{1.5em} %sangria

% Thicker lines in tables
\makeatletter
\newcommand{\thickhline}{%
	\noalign {\ifnum 0=`}\fi \hrule height 1pt
	\futurelet \reserved@a \@xhline
}
\makeatother

% Subsubsubsection (|paragraph)
\setcounter{tocdepth}{4}
\setcounter{secnumdepth}{4}

\begin{document}	
	
	\maketitle 
	\newpage
	\tableofcontents
	\newpage
	
	\section{Multiple Linear Regression}
	
	\subsection{Introduction}
	
	In this first section, we will conduct a multiple linear regression following question 1 of the 5th homework assignment. We will estimate the regression coefficients (betas) following different several methods seen during the lectures and we will provide an estimation for a new response.
	
	\subsection{Summary}
	
	For this first analysis, I've selected the Battery Failure dataset. In this example, we want to predict the cycles of life of a certain battery before it fails. We are provided the following variables: Charge rate (amps), discharge rate (amps), depth of discharge (\% of rated ampere-hours), temperature (ºC), and end of charge voltage (volts). These are the first three rows of data: 
	
	\begin{table}[H] \centering
		\begin{tabular}{cccccc}
			$Z_1$                                                         & $Z\_2$                                                             & $Z\_3$                                                                                           & $Z\_4$                                                        & $Z\_5$                                                                           & Y                                                            \\
			\begin{tabular}[c]{@{}c@{}}Charge rate \\ (amps)\end{tabular} & \begin{tabular}[c]{@{}c@{}}Discharge Rate\\  (amps)\end{tabular} & \begin{tabular}[c]{@{}c@{}}Depth of \\ Discharge \\ (\% of rated \\ ampere-hours)\end{tabular} & \begin{tabular}[c]{@{}c@{}}Temperature\\  (ºC)\end{tabular} & \begin{tabular}[c]{@{}c@{}}End of \\ Charge \\ Voltage \\ (volts)\end{tabular} & \begin{tabular}[c]{@{}c@{}}Cicles to \\ failure\end{tabular} \\ \hline
			0.375                                                         & 3.13                                                             & 60.0                                                                                           & 40                                                          & 2.00                                                                           & 101                                                          \\
			1.000                                                         & 3.13                                                             & 76.8                                                                                           & 30                                                          & 1.99                                                                           & 141                                                          \\
			1.000                                                         & 3.13                                                             & 60.0                                                                                           & 20                                                          & 2.00                                                                           & 96                                                          
		\end{tabular}
	\end{table}
	
	\subsection{Analysis}
	
	\textbf{(1) Find the least square estimate beta hat} \\
	
	We obtain the least square estimate beta hat following out notes:
	
	$$ \hat{\vec{\beta}} = (Z^T \cdot Z)^{-1} \cdot Z^T \cdot \vec{Y} $$
	
	Obtaining:
	$$ \hat{\vec{\beta}} =
		\begin{pmatrix}
		-2937.7571 \\
		-33.7934   \\
		-0.1798   \\
		-1.7397    \\
		7.0627     \\
		1529.2897
		\end{pmatrix}
	$$
	
	\textbf{(2) Find the $R^2$ statistic} \\
	
	We use:
	
	$$ R^2 = \frac{||\hat{\vec{Y}} - \bar{Y} \cdot \vec{1_n}||^2}{||\vec{Y} - \bar{Y} \cdot \vec{1_n}||^2} $$
	
	Obtaining $R^2 = 0.4799$. \\
	
	\textbf{(3) Find sigma\_hat\_square and estimated Cov(beta square)} \\
	
	We use:
	
	$$ \hat{\sigma}^2 = \frac{1}{n-r-1} ||\hat{\vec{\epsilon}}||^2 $$
	
	and
	
	$$ \hat{Cov}(\hat{\vec{\beta}}) = \hat{\sigma}^2 (Z^T Z)^{-1} $$
	
	We obtain the following:
	
	$$ \hat{\sigma}^2 = 7138.186 $$	
	$$ \hat{Cov}(\hat{\vec{\beta}}) = 
	\begin{pmatrix}
	1.633e+07  & -2933.74 & 2980.4460  & -34.78143 & -991.73637 & -8.160e+06 \\
	-2.934e+03 & 1880.55  & 18.5503    & 17.34897  & 10.28445   & -1.764e+02 \\
	2.980e+03  & 18.55    & 193.4117   & -3.23257  & 0.34449    & -1.696e+03 \\
	-3.478e+01 & 17.35    & -3.2326    & 1.79944   & -0.08092   & -4.242e+01 \\
	-9.917e+02 & 10.28    & 0.3445     & -0.08092  & 3.89193    & 4.549e+02  \\
	-8.160e+06 & -176.39  & -1695.7845 & -42.42251 & 454.86652  & 4.081e+06 
	\end{pmatrix}
	$$ \\
	
	\textbf{(4) 95\% confidence interval for each $\beta_j$} \\
	
	We use one at a time confidence intervals for the betas:
	
	$$ \beta_j \in [ \hat{\beta_j} \pm \hat{\sigma} \cdot \sqrt{\omega_{jj}} \cdot t_{n-r-1}(\frac{\alpha}{2}) ] $$
	
	Obtaining:
	
	\begin{table}[H] \centering
		\begin{tabular}{l}
			$\beta_0 \in [ -11604 , 5729 ]$ \\
			$\beta_1 \in [ -126.8 , 59.22 ]$ \\
			$\beta_2 \in [ -30.01 , 29.65 ]$ \\
			$\beta_3 \in [ -4.617 , 1.137 ]$ \\
			$\beta_4 \in [ 2.831 , 11.29 ]$ \\
			$\beta_5 \in [ -2804 , 5862 ]$
		\end{tabular}
	\end{table}
	
	\textbf{(5) 95\% simultaneous confidence intervals for all betas based on the confidence region} \\
	
	Using the formula from the notes:
	
	$$ \beta_j \in [ \hat{\beta_j} \pm \hat{\sigma} \cdot \sqrt{\omega_{jj}} \cdot \sqrt{(r+1) \cdot F_{r+1,n-r-1}(\alpha)} ] $$
	
	We obtain:
	
	\begin{table}[H] \centering
		\begin{tabular}{l}
			$\beta_0 \in [ -19640 , 13764 ]$ \\
			$\beta_1 \in [ -213 , 145.5 ]$ \\
			$\beta_2 \in [ -57.67 , 57.31 ]$ \\
			$\beta_3 \in [ -7.285 , 3.805 ]$ \\
			$\beta_4 \in [ -1.092 , 15.22 ]$ \\
			$\beta_5 \in [ -6822 , 9880 ]$
		\end{tabular}
	\end{table}
	
	\textbf{(6) 95\% simultaneous confidence intervals for all betas based on the Bonferroni correction} \\
	
	We compute a final set of intervals for the ebtas using the Bonferroni correction:
	
	$$ \beta_j \in [ \hat{\beta_j} \pm \hat{\sigma} \cdot \sqrt{\omega_{jj}} \cdot t_{n-r-1}(\frac{\alpha}{2(r+1)}) ] $$
	
	We obtain:
	
	\begin{table}[H] \centering
		\begin{tabular}{l}
			$\beta_0 \in [ -15338 , 9462 ]$ \\
			$\beta_1 \in [ -166.9 , 99.29 ]$ \\
			$\beta_2 \in [ -42.86 , 42.5 ]$ \\
			$\beta_3 \in [ -5.856 , 2.377 ]$ \\
			$\beta_4 \in [ 1.009 , 13.12 ]$ \\
			$\beta_5 \in [ -4670 , 7729 ]$
		\end{tabular}
	\end{table}
	
	\textbf{(7) Test $H_0: \beta_1 = \beta_2 = 0 $ at significance level $\alpha = 0.05$} \\
	
	Using this matrix for the linear transformation:
	
	$$ C = 
	\begin{pmatrix}
		0 & 1 & 0 & 0 & 0 & 0 \\
		0 & 0 & 1 & 0 & 0 & 0 \\
		0 & 0 & 0 & 1 & 0 & 0 \\
		0 & 0 & 0 & 0 & 1 & 0 \\
		0 & 0 & 0 & 0 & 0 & 1
	\end{pmatrix}
	$$
	
	We can compute the F-test statistic:
	
	$$ \vec{\beta}^T_{(2)} \Omega_{22}^{-1} \vec{\beta}_{(2)} = 108296 $$
	
	And compare it to:
	
	$$ (r-q) \cdot \hat{\sigma}^2 \cdot F_{r-q,n-r-1}(\alpha) = 133445 $$
	
	Since $108296 < 133445$, we don't have sufficient evidence to assure that $\beta_1 = \beta_2 = 0$. \\
	
	\textbf{(8) 95\% confidence interval for the mean response given $\mathds{E}(Y_0) = \beta_0 + \sum_{i=1}^{5} \beta_i \cdot \bar{z}_i$, where $\bar{z}_i$ is the sample mean of $z_{i,j}$ for $i \in \{1, ..., n\}$} \\
	
	First, compute $\vec{z_0}$:
	
	$$ \vec{z_0} = 
	\begin{pmatrix}
		1.000  \\
		1.031  \\
		3.034  \\
		62.840 \\
		19.500 \\
		1.999 
	\end{pmatrix} 
	$$
	
	And now compute the confidence intervals for it's associated value using the formula in the class notes:
	
	$$ \vec{z_0}^T \vec{\beta} \in [\vec{z_0}^T \hat{\vec{\beta}} \pm \hat{\sigma} \cdot t_{n-r-1}(\frac{\alpha}{2}) \sqrt{\vec{z_0}^T (Z^T Z)^{-1} \vec{z_0}} ] $$
	
	Obtaining the following interval: $ \vec{z_0}^T \vec{\beta} \in [ 71.78 , 152.8 ] $ \\
	
	\textbf{(9) 95\% confidence interval for a new response $Y_0$ given $\vec{z_0}$} \\
	
	Using a similiar formula: 
	
	$$ \vec{z_0}^T \vec{\beta} \in [\vec{z_0}^T \hat{\vec{\beta}} \pm \hat{\sigma} \cdot t_{n-r-1}(\frac{\alpha}{2}) \sqrt{1 + \vec{z_0}^T (Z^T Z)^{-1} \vec{z_0}} ] $$
	
	And using that $Y_0 = \vec{z_0}^T \vec{\beta} + \epsilon_0$ we obtain:
	
	$$ Y_0 \in [ -73.38 , 298 ] $$
	
	A substantialy bigger interval than the over in \textbf{(8)}, which makes sense since we are including the error now.
	
	\section{Two-Sample test and LDA}
	
	\subsection{Introduction}
	Briefly summarize the goal of the analysis in your own words
	\subsection{Summary}
	Summarize your data by plots or sample estimates
	\subsection{Analysis}
	Implement the analysis based on what you have done in homework
	
	\section{PCA}
	
	\subsection{Introduction}
	Briefly summarize the goal of the analysis in your own words
	\subsection{Summary}
	Summarize your data by plots or sample estimates
	\subsection{Analysis}
	Implement the analysis based on what you have done in homework
	
\end{document}